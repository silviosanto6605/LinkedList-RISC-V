\documentclass[10pt]{article}
\usepackage[inkscapelatex=false]{svg}
\usepackage{geometry}
\geometry{margin=1.5in}
\usepackage{subfig}
\usepackage{graphicx} 
\usepackage{nopageno}
\title{Relazione progetto finale Architetture degli Elaboratori - A.A. 2024/2025}

\date{}

\begin{document}
   
    \maketitle
    \begin{center}
        {\includesvg[width=0.75\columnwidth]{img/unifi.svg}}
    \end{center}

    \begin{center}
        \vspace{9cm}
        \author{Silvio Santoriello\\ \textit{silvio.santoriello@edu.unifi.it}
        \\ Mat. 7158636
        }
    \end{center}
    \pagestyle{plain}
    
    \newpage
    
    \section{Main loop e parsing della stringa}
    Secondo le specifiche del progetto, l'input viene passato tramite stringa \textit{listInput} all'interno della sezione \textit{.data}. e ogni comando ben formattato è separato da $\sim$ (carattere ASCII 126).

    Il main loop trasferisce immediatamente il controllo alla funzione di PARSING, che analizzerà carattere per carattere il \textit{listInput}.\\ \\

    \subsection{PARSING}

        Tale procedura consiste di un ciclo principale, chiamato \textit{parsing\_loop} che controlla innanzitutto la presenza di spazio o tilde e nel caso li ignora, proseguendo la scansione.
        \\Ho preferito dividere la fase di parsing in più parti:
        \begin{itemize}
            \item[$\diamond$] identificazione del comando (tramite \textit{identify\_command})
            \item[$\diamond$] validazione del formato del comando (tramite \textit{process\_\textbf{x}\_command} e \textit{verify\_\textbf{x}\_format)}
            \item[$\diamond$] esecuzione/non esecuzione sulla base del risultato della validazione
        \end{itemize}

        Il comando viene identificato tramite un serie di controlli annidati, riassunti in parte dal seguente schema: \\ \\
        \begin{figure}
            \centering
            \includesvg[width=0.5\linewidth]{img/identify_command_chart.svg}
            \caption{Caption}
            \label{fig:enter-label}
        \end{figure}

        
    
        
        
    

    

\end{document}
